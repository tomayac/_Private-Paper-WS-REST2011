% THIS IS SIGPROC-SP.TEX - VERSION 3.1
% WORKS WITH V3.2SP OF ACM_PROC_ARTICLE-SP.CLS
% APRIL 2009
%
% It is an example file showing how to use the 'acm_proc_article-sp.cls' V3.2SP
% LaTeX2e document class file for Conference Proceedings submissions.
% ----------------------------------------------------------------------------------------------------------------
% This .tex file (and associated .cls V3.2SP) *DOES NOT* produce:
%       1) The Permission Statement
%       2) The Conference (location) Info information
%       3) The Copyright Line with ACM data
%       4) Page numbering
% ---------------------------------------------------------------------------------------------------------------
% It is an example which *does* use the .bib file (from which the .bbl file
% is produced).
% REMEMBER HOWEVER: After having produced the .bbl file,
% and prior to final submission,
% you need to 'insert'  your .bbl file into your source .tex file so as to provide
% ONE 'self-contained' source file.
%
% Questions regarding SIGS should be sent to
% Adrienne Griscti ---> griscti@acm.org
%
% Questions/suggestions regarding the guidelines, .tex and .cls files, etc. to
% Gerald Murray ---> murray@hq.acm.org
%
% For tracking purposes - this is V3.1SP - APRIL 2009

\documentclass{acm_proc_article-sp}

%\usepackage{url}
\usepackage[hyphens]{url}
\usepackage{textcomp}

\begin{document}

\title{Enabling HATEOAS Through HTTP OPTIONS, Link Elements, And The HTTP Vocabulary In RDF}
%
% You need the command \numberofauthors to handle the 'placement
% and alignment' of the authors beneath the title.
%
% For aesthetic reasons, we recommend 'three authors at a time'
% i.e. three 'name/affiliation blocks' be placed beneath the title.
%
% NOTE: You are NOT restricted in how many 'rows' of
% "name/affiliations" may appear. We just ask that you restrict
% the number of 'columns' to three.
%
% Because of the available 'opening page real-estate'
% we ask you to refrain from putting more than six authors
% (two rows with three columns) beneath the article title.
% More than six makes the first-page appear very cluttered indeed.
%
% Use the \alignauthor commands to handle the names
% and affiliations for an 'aesthetic maximum' of six authors.
% Add names, affiliations, addresses for
% the seventh etc. author(s) as the argument for the
% \additionalauthors command.
% These 'additional authors' will be output/set for you
% without further effort on your part as the last section in
% the body of your article BEFORE References or any Appendices.

\numberofauthors{2} %  in this sample file, there are a *total*
% of EIGHT authors. SIX appear on the 'first-page' (for formatting
% reasons) and the remaining two appear in the \additionalauthors section.
%
\author{
% You can go ahead and credit any number of authors here,
% e.g. one 'row of three' or two rows (consisting of one row of three
% and a second row of one, two or three).
%
% The command \alignauthor (no curly braces needed) should
% precede each author name, affiliation/snail-mail address and
% e-mail address. Additionally, tag each line of
% affiliation/address with \affaddr, and tag the
% e-mail address with \email.
%
% 1st. author
\alignauthor Thomas Steiner\\
       \affaddr{Universitat Polit{\'e}cnica de Catalunya}\\
       \affaddr{Department LSI}\\
       \affaddr{08034 Barcelona, Spain}\\
       \email{tsteiner@lsi.upc.edu}
\alignauthor Jan Algermissen\\
       \affaddr{NORD Software Consulting}\\
       \affaddr{Kriemhildstrasse 7}\\
       \affaddr{22559 Hamburg}\\
       \email{info@nordsc.com}
}
% Just remember to make sure that the TOTAL number of authors
% is the number that will appear on the first page PLUS the
% number that will appear in the \additionalauthors section.

\maketitle
\begin{abstract}
\end{abstract}

% A category with the (minimum) three required fields
\category{H.3}{Information Storage and Retrieval}{On-line Information Services}

\terms{Experimentation}

\keywords{RDF, LOD, Linked Data, Semantic Web, NLP, Video} % NOT required for Proceedings

\section{Introduction}\label{sec:introduction}
With SemWebVid \cite{Steiner:SemWebVid} we introduced a client-side interactive Ajax application for the automatic generation of RDF video annotations. For this paper we have re-implemented and vastly improved the annotation logic on the server-side, resulting in a RESTful read/write-enabled Web service for RDF video annotations. A YouTube video is described by a Google Data Atom feed\footnote{E.g., \url{http://gdata.youtube.com/feeds/api/videos/Rq1dow1vTHY}}. In order to semantically annotate the various elements of this feed, we concentrated on the following fields (in XPath syntax): title \texttt{/entry/media:group/media:title}, description \texttt{/entry/\-media:\-group/media:description}, tags \texttt{/entry/media:\-group/media:\-keywords}. YouTube offers an automatic audio transcription service and users can also upload audio transcriptions on their own. This allows for closed captions in several languages (we differentiate between subtitles and closed captions, where subtitles are hard-encoded into the video, and closed captions separate resources). In addition to the previously mentioned elements of the Google Data Atom feed, we thus use closed captions\footnote{E.g., \url{http://www.youtube.com/watch_ajax?action_get_caption_track_all&v=Rq1dow1vTHY}} when they are available.

The remainder of this paper is structured as follows: 

\section{Conclusion}\label{sec:conclusion}

%\end{document}  % This is where a 'short' article might terminate

%ACKNOWLEDGMENTS are optional
\section{Acknowledgments}\label{sec:acknowledgments}
This work is partly funded by the EU FP7 I-SEARCH project (project reference 248296).

%
% The following two commands are all you need in the
% initial runs of your .tex file to
% produce the bibliography for the citations in your paper.
\bibliographystyle{abbrv}
\bibliography{sigproc-sp}  % sigproc.bib is the name of the Bibliography in this case
% You must have a proper ".bib" file
%  and remember to run:
% latex bibtex latex latex
% to resolve all references
%
% ACM needs 'a single self-contained file'!
%
%APPENDICES are optional

%\appendix
%Appendix A
%\balancecolumns
\end{document}
